\documentclass{article}
\usepackage{listings}
\usepackage[table]{xcolor}
\usepackage{varioref}
\usepackage{float}
\usepackage{graphicx}
\usepackage{geometry} %gives less margins automagically
\usepackage{gensymb} %°
\begin{document}
\lstset{basicstyle=\ttfamily} % fixed-width listings

\definecolor{lgray}{gray}{0.95}
\definecolor{dgray}{gray}{0.90}

\title{KUB manual}
\date{\today}
\author{Tomas H\"ardin}

\maketitle

\newpage

\tableofcontents
\listoffigures
\listoftables

\newpage

\section{Assembly}


The instrument consists of a stainless steel skeleton into which a number of aluminium plates are screwed.
All plates have a printed circuit board (PCB) attached via standoffs or screwed directly to the plate.
The combination of board and plate is called a module.
In this section we assume all PCBs have already been soldered and cleaned.

The following tools and materials are needed:

\begin{itemize}
\item Wire cutters
\item Wire strippers suitable for 0.4 mm and 0.9 mm diameter wire (26 and 20 AWG). Strippers with fixed holes are recommended
\item A small round or semi-round file
\item Torque wrenches capable of 0.3 Nm, 0.8 Nm and 2.0 Nm with the following bits:
\begin{itemize}
\item PH1 Phillips
\item 9IP Torx-Plus or T10 Torx
\item 4.5 mm hex socket
\item 5.5 mm hex socket
\item 7.0 mm hex socket
\end{itemize}
\item Loctite 638
\item Scotch-Weld 2216
\item Soldering station
\item Helping hands
\item 60/40 or 63/37 leaded solder
\item Solder flux
\item White gloves for handling silver
\item Two fine paintbrushes
\item Citadel Chaos Black primer
\item The Army Painter WP1101 Matt Black
\item Electrolube SCP03G Silver Conductive Paint
\item Toothpicks
\item Heat gun
\item 0.62 $mm^2$ / 20 AWG Alpha Wire 5856 WH005 PTFE wire
\item 2.4 mm $\rightarrow$ 1.2 mm Kynar shrink tube
\item Masking tape
\item Maxon EC motor can vice (3D printed)
\end{itemize}

Each PCB can be screwed to its corresponding aluminium plate in any order, forming modules.
The modules must be assembled in a certain order however,
due to the way the connectors fit into eachother.
The power4 module must be mounted first,
then the cpu4 and credits modules can be mounted, and finally the fieldmill10 modules can be mounted.
It's useful to mount the credits module last so that the mating of the fieldmill10 and cpu4 modules can be checked,
and power-on tests performed.

In order to figure out what ID each DS18B20Z 1-wire temperature sensor has,
the instrument should be powered on after each module has been installed and
a temperature measurement performed (the 't' command in table \ref{command_table_0}).
Since the cpu4 and power4 modules must be mounted at the same time in order to use the RS-485 interface,
the two temperature sensors in that assembly must be distinguished somehow.
This can easily done by warming one of them with a finger and watching the rise in temperature in one of the sensors.
Take note of which sensor is which.
This goes for all modules.

Use white cotton gloves when handling any silver parts, to prevent them from being contaminated.
Each aluminium plate requires 16 silver plates M3 screws (?? mm long, Torx T10) to mount it to the skeleton, 96 total.

All M3 screws are torqued to 0.8 Nm and all M2 screws are torqued to 0.3 Nm.

\subsection{power4}

Parts needed:

\begin{itemize}
\item One (1) power4 PCB, soldered and cleaned
\item One (1) silver plated bottom aluminium plate (2mm thick)
\item Two (2) power4 aluminium standoffs
\item Four (4) PEEK washers (top kind)
\item Four (4) PEEK washers (bottom kind)
\item Twentyfour (24) silver plated M3 screws, ?? length inluding head (Torx T10)
\item Four (4) pieces of Alpha Wire 5856 WH005 PTFE wire (TODO: length)
\item 2.4 mm $\rightarrow$ 1.2 mm Kynar shrink tube
\item One (1) DE-9 male connector
\item A set of DE-9 panel mounting screws, washers and nuts
\end{itemize}

If soldering the power4 module yourself,
the pin headers should be inserted from the same side as the DS18B20Z IC and LEDs are on.

Cut four pieces of the PTFE wire to appropriate lengths (TODO: include in table),
then solder them between the power4 board the the DE-9 connector according to table \ref{power4_table}.
It is easiest to solder the +28V wire last, since it crosses the other three wires,
see figure \ref{power4_wires}.
The pins on the power4 board are numbered from 1 starting from the side closest to cpu4.
There are two columns of identical pins,
the purpose of which is so that PTFE wire stripped around 15 mm can be threaded through and soldered to both holes,
providing increased rigidity.
Use Kynar shrink tubing to protect the solder points on the DE-9 side.

\begin{table}[H]
\begin{centering}
\rowcolors{1}{lgray}{dgray}
\begin{tabular}{|l|l|l|}
\hline
{\bf Description} & {\bf DE-9 pin}    & {\bf power4 pin}\\ \hline
+28V              & 1                 & 1 \\ \hline
GND               & 5                 & 2 \\ \hline
RS-485+ A         & 3                 & 3 \\ \hline
RS-485- B         & 7                 & 4 \\ \hline
\end{tabular}
\caption{power4 wiring.}
\label{power4_table}
\end{centering}
\end{table}


\begin{figure}
\centering
% Cut down on PDF size in git for now
%\includegraphics[width=12cm]{power4}
\caption{Photograph of power4 wiring.
TODO: this should show the weaving of wire through both columns of pins.}
\label{power4_wires}
\end{figure}

Fasten the DE-9 connector to the aluminium plate (inside or outside??) using the screws, washers and nuts.
Torque to 0.8 Nm.
Screw the standoffs to the plate, 0.8 Nm.

Place bottom PEEK washers on the standoffs, then the power4 board on the standoffs and finally the top wasters on top.
Screw everything in place using M3 screws, 0.8 Nm.

Screw the finished module to the skeleton using M3 screws, 0.8 Nm.

\subsection{cpu4}

Parts needed:

\begin{itemize}
\item One (1) cpu4 PCB, soldered and cleaned
\item One (1) silver plated cpu4 aluminium plate
\item Two (2) cpu4/credits aluminium standoffs
\item Four (4) PEEK washers (top kind)
\item Four (4) PEEK washers (bottom kind)
\item Twentyfour (24) silver plated M3 screws, ?? length inluding head (Torx T10)
\item A Debian or Ubuntu GNU/Linux computer with build-essentials, avr-gcc and avrdude installed
\item An STK600
\item A USB to RS-485 converter
\item A lab power supply
\item Instrument converter cable (female DE-9 instrument connector $\rightarrow$ female RS-485 DE-9 connector(s) + banana plugs)
\item One or two 2920 size polyfuses (1 or 2 A)
\end{itemize}

Make sure there's 2 Amperes worth of polyfuses installed in F1.
The boards we have now are 1 A, which has been slightly too little in some tests when all motors are heavily loaded, despite the specs on the motors and the 24-to-24 V regulator.

Be careful to use the aluminium plate with holes countersunk in the correct direction.
The credits plates are mirror images of the cpu4 plates.

Screw the standoffs to the plate using M3 screws, 0.8 Nm.

Places bottom PEEK washers, board and top PEEK washers on the standoffs.
Screw in place using M3 screws, 0.8 Nm.

Screw the finished module to the skeleton using M3 screws, 0.8 Nm.

\subsubsection{Programming the bootloader}

The bootloader can be programmed without the power4 board.
Connect the cpu4 board to an STK600 using a 6-pin ISP (2x3 ribbon) cable,
then connect the STK600 to the computer and turn the STK600 on.
Run \emph{make bootloader} in the code directory to build and program the bootloader.
This will enable the crystal oscillator, and programming over RS-485.

After programming, make sure the crystal oscillator is working correctly by measuring the waveform on both its pins.
The pin closest to the ISP connector should swing between $+0.28 \dots +2.76$ V, $\pm 100$ mV or so.
This pin is what drives the ADCs' clocks.
The pin further away from the ISP connector should have a larger swing, between $-0.32 \dots +3.72$ V, $\pm 100$ mV or so.
Both waveforms should be roughly sinusodial.

\subsubsection{Serial programming}

The next step is to program the application via the RS-485 bus.
To do this, mount the power4 and cpu4 modules to the skeleton and connect the instrument converter cable to the instrument.
Connect the USB to RS-485 converter between the computer and the RS-485 end of the instrument cable.
Set the power supply to somewhere between $18 \dots 30$ V,
disable its output and connect the banana plugs to it (mind the polarity).
When you enable the lab supply's output, you have three seconds before the bootloader enters the application automatically.
Type \emph{make programserial} into a terminal in the code directory, but don't hit enter yet.
When you are ready, enable the lab supply's output and then hit enter in the terminal.
If everything went OK then the instrument should now have the application loaded into it.
You can now proceed to checking system voltages!

\subsubsection{Checking system voltages}

After programming the application,
the MCU will enable the $\pm 5$ V and $24$ V switching regulators after booting (3 seconds after power-up).
Once it has done so, you should measure all system voltages to make sure they are reasonable.
See table \ref{system_voltages}.

\begin{table}[H]
\begin{centering}
\rowcolors{1}{lgray}{dgray}
\begin{tabular}{|l|l|l|}
\hline
{\bf Voltage bus} & {\bf Min}    & {\bf Max}\\ \hline
+3.3V             & $+3.2$ V     & $+3.4$ V \\ \hline
+24V              & $+23$  V     & $+25$  V \\ \hline
+5V               & $+4.9$ V     & $+5.1$ V \\ \hline
-5V               & $-5.1$ V     & $-4.9$ V \\ \hline
\end{tabular}
\caption{System voltage ranges.}
\label{system_voltages}
\end{centering}
\end{table}

You should also measure the resistances and voltages across the system voltage dividers
(R20 through R27, R37 and R38), for calibrating the application.
After such calibration, issue a 'v' command (table \ref{command_table_0}),
to ensure that the MCU's idea of what the system voltages are, is in line with the actual measured voltages.

Next up is distinguishing the two temperature sensors now in the system.

\subsubsection{Distinguishing temperature sensors}

Distinguish the two DS18B20Z temperature sensors now in the instrument by touching one of them, thus warming it, then issue a 't' command (table \ref{command_table_0}).
Write down which ID corresponds to which of the two sensors.

\subsection{fieldmill10}

Parts needed:

\begin{itemize}
\item One (1) fieldmill10 PCB
\item One (1) fieldmill aluminium top plate
\item One (1) fieldmill shutter plate
\item One (1) Maxon 349694 EC motor
\item Masking tape
\item One (1) ITR20001/T IR reflex coupler
\item One (1) 0805 4.7 $k\Omega$ resistor
\item Three (3) Würth 36903205S 20x20x3 mm shielding cans.
\item Three (3) M2 screws, 6 mm length including head (Phillips PH1)
\item Four (4) silver plated M3 screws, 10 mm length including head (Torx T10)
\item Sixteen (16) silver plated M3 screws. Same as above??
\item Four (4) M3 washers
\item Four (4) M3 hex nuts
\item One (1) M4 X 8/8 hex screw with 2 mm hole drilled through, preferably silver plated or silver painted if {\it someone} forgot to order silver plated M4 screws
\item One (1) silver plated M4 flange nut
\end{itemize}

Before doing anything else, perform the following electrical tests using a multimeter

\begin{itemize}
\item Measure resistance from GND to every rail; +3.3V, +24V, +5V, -5V, +2.5V and -2.5V. Measure in both directions. Values should all be at least 1 k$\Omega$
\item Measure resistance from either side of R56 to GND. Measure in both directions. Value should be at least 8 k$\Omega$
\item Measure the resistance between AINP4 (where R3 and R5 connect) and GND. Value should be between 9.90 k$\Omega$ and 10.10 k$\Omega$
\end{itemize}

Applying power while these values are out of spec may cause PERMANENT DAMAGE to op-amps, DACs and/or linear regulators.
In the worst case the +-5V regulator (TMR 6-2421WI) may also become damaged.
Once checked and within spec the assembly process may continue.

\begin{table}[H]
\begin{centering}
\rowcolors{1}{lgray}{dgray}
\begin{tabular}{|l|l|l|}
\hline
{\bf fm10 voltage bus} & {\bf Min}    & {\bf Max} \\ \hline
+2.5V                  & $+2.45$ V    & $+2.55$ V \\ \hline
-2.5V                  & $-2.55$ V    & $-2.45$ V \\ \hline
\end{tabular}
\caption{fieldmill10 voltage ranges.}
\label{fm10_voltages}
\end{centering}
\end{table}

Cut motor wires to 6-7 mm length.
Strip so that 3 mm of insulation remains, with the wire stripper set to 0.4 mm.
Twist and tin the ends of the wires. See figure \vref{motorwires}.

\begin{figure}
\centering
% Cut down on PDF size in git for now
%\includegraphics[width=12cm]{3mm7mm}
\caption{Motor wires soldered to their pads. Wire and insulation lengths marked.}
\label{motorwires}
\end{figure}

\begin{figure}
\centering
% Cut down on PDF size in git for now
%\includegraphics[width=12cm]{motors}
\caption{Motors with rotor cans painted black 180$\degree$ around}
\label{motors}
\end{figure}


Mask off half the motor can using masking tape and paint the exposed area matt black.
One way to do this is to spray primer into a glass or glazed ceramic cup then dip a fine paintbrush into the primer and paint it onto the can.
DO NOT USE A PLASTIC CUP TO HOLD THE PRIMER - IT WILL MELT.
Wait at least 30 minutes for the primer to dry, then apply the matt black paint on top of it.
The black painted area should cover a 180$\degree$ arc of the of the can, especially the edge.
See figure \vref{motors}.

\begin{figure}
\centering
% Cut down on PDF size in git for now
%\includegraphics[width=12cm]{reflectometer}
\caption{Reflex coupler and motor can. Black paint also visible on motor can (earlier 90$\degree$ arc)}
\label{reflectometer}
\end{figure}

If the motor doesn't fit in the central hole in the PCB, use a small semi-round file
to grind away some of the PCB material until the motor fits.

Put the Maxon EC motor in the motor hole,
then shim the PCB + motor combination up from the bottom so that
the motor can lie flush with the PCB while working on it.
Don't solder the motor in just yet, as that will make mounting the reflex coupler and shielding cans more difficult.

Bend and cut the IR reflex coupler leads so that the reflex coupler looks directly at the EC motor's rotor can when inserted into its 2x3 socket (IR2 reference on PCB). See figure \vref{reflectometer}.
Bending then trimming all leads to the same length as the shortest lead will accomplish this,
The resulting lead length should be 7 mm from the bottom of the reflex coupler.
Insert the reflex coupler into its socket.
The gap between the top of the 2x3 socket and the bottom of the reflex coupler should be 2 - 3 mm.
Remove the motor, then solder the reflex coupler into the socket.
Use plenty of flux so the leads and socket are guaranteed to be soldered together.
Some extra solder can be applied to the 2x3 socket pads at this point.
Carefully test that pulling on the reflex coupler doesn't cause it to come out.
Check continuity with a multimeter, and that adjacent pins aren't shorted.

Replace R56 with a 4.7 $k\Omega$ 0805 resistor.

Solder the shielding cans in place over each of the three analog frontends.

Reinstall the motor and solder the five motor wires to the motor wire pads on the PCB.

Screw the motor + PCB assembly to the aluminium plate using the M2 screws for the motor and silver plated M3 screws, washers and nuts for the PCB.
First screw in the screws lightly, then tighten using the torque wrench. Use 0.3 Nm for the M2 screws and 0.8 Nm for the M3 nuts.

Use a toothpick to paint the motor bearing with SCP03G silver paint.
Be careful not to get silver paint on the motor axis.
There should be a slight resistance in the bearings which will go away after the first initial minutes of operation.

Use Loctite 638 to glue the M4 screw to the motor axis.
Mount the screw so that the head points downward, toward the instrument,
and so that it is flush with the inner ring of the motor bearing.
When the glue has dried, put the motor can in the specially shaped vice.
Put the shutter plate on the M4 screw and fasten it using the M4 flange nut.
Torque to 2.0 Nm.
Drop some SCP03G silver paint into the top of the screw so that it gets a good electrical connection to the motor shaft.

Screw each module to the skeleton using M3 screws, 0.8 Nm.

Perform a power-on test.
Check that the +5V, -5V, +3.3V, +2.5V and -2.5V rails are within spec (tables \ref{system_voltages} and \ref{fm10_voltages}).
Measure the output of every channel while VGND = 0 V. The average voltage of each output should be within $\pm$200 mV of VGND.

Take a temperature reading after each fieldmill10 module has been inserted,
and make a note of the new sensor ID each time and what module it corresponds to.

Start the motors and run them for at least five minutes.
Measure the resistance between motor axis and instrument ground.
The resistance should be less than 10 $\Omega$.

\subsection{credits}

\begin{itemize}
\item One (1) credits PCB, soldered and cleaned
\item One (1) silver plated credits aluminium plate
\item Two (2) cpu4/credits aluminium standoffs
\item Four (4) PEEK washers (top kind)
\item Four (4) PEEK washers (bottom kind)
\item Twentyfour (24) silver plated M3 screws, ?? length inluding head (Torx T10)
\end{itemize}

Be careful to use the aluminium plate with holes countersunk in the correct direction.
The credits plates are mirror images of the cpu4 plates.

Screw the standoffs to the plate using M3 screws, 0.8 Nm.

Places bottom PEEK washers, board and top PEEK washers on the standoffs.
Screw in place using M3 screws, 0.8 Nm.

Screw the finished module to the skeleton using M3 screws, 0.8 Nm.

Perform a power-on test and a temperature test.
Take note of the new temperature sensor ID.

\newpage
\section{Command and response summary}

All commands are human-readable, start with a single ASCII character and are terminated with a line ending.
Comments may be added by inserting a hash sign ('\#').
This causes the rest of the line to be ignored (characters are consumed and discarded until end-of-line).
Mistakes can be corrected with backspace (BS, ASCII code 8) or delete (DEL, ASCII code 127), both of which are treated as a backspace.
Line reading also understands escape (ESC, ASCII code 27), which aborts the current command.

Parameter parsing is handled by sscanf(), which allows for the same command character to take a varying number
of parameters. An example of this is the 'M' command which exists in two-parameter and three-parameter forms:

\begin{lstlisting}
    M0 10       # Set speed of motor 0 to 10
    M10 10 10   # Set speed of all motors to 10
\end{lstlisting}


Line endings can either be carriage return ('{\textbackslash}r', ASCII code 13) or linefeed ('{\textbackslash}n', ASCII code 10), but never both in the same line.
In other words both Unix and Mac line endings are OK, but Windows line endings ("{\textbackslash}r{\textbackslash}n") are not.
This ensures that {\it echo}, {\it minicom} and {\it picocom} work as expected.
Output from the instrument is however terminated by Windows line endings ("{\textbackslash}r{\textbackslash}n"), in order to play nice with {\it minicom} and {\it picocom}.
An ESC ('{\textbackslash}x1B', ASCII code 27) anywhere in a line aborts that command.
The instrument will respond to ESC immediately.
There is no need to terminate ESC with newline.

Output from the instrument is delimited into frames.
A frame starts with the string "BUSY{\textbackslash}r{\textbackslash}n" and ends with the string "READY{\textbackslash}r{\textbackslash}n".
Each frame contains one or more sections, each of which is started by a star (*) followed by the name of the section terminated by {\textbackslash}r{\textbackslash}n.
For example, the reply to the "M10 10 10" command example given earlier is of type MTR\_PWM and would therefore be:

\begin{lstlisting}
    BUSY
    *MTR_PWM
    10 10 10
    READY
\end{lstlisting}

Tables \ref{command_table_0} and \ref{command_table} summarize all commands and their parameters.
More detailed descriptions of each command and response are given in the sections that follow.

\begin{table}[H]
\begin{centering}
\rowcolors{1}{lgray}{dgray}
\begin{tabular}{|p{1.0cm}|p{7.8cm}|p{2.0cm}|}
\hline
{\bf Char} & {\bf Description}    & {\bf Response}\\ \hline
v & Measure system voltages       & VOLTAGES      \\ \hline
V & Enable 24V and +-5V           & VOLTAGES      \\ \hline
B & Disable 24V and +-5V          & VOLTAGES      \\ \hline
m & Read motor PWMs               & MTR\_PWM      \\ \hline
K & Set motor PWMs to 50\%        & MTR\_PWM      \\ \hline
o & Read VGND DAC values          & VGNDS         \\ \hline
r & Measure motor speeds in RPM   & MTR\_SPD      \\ \hline
1 & List 1-wire device ROMs       & ONEWIRE       \\ \hline
! & Search for 1-wire devices     & ONEWIRE       \\ \hline
t & Measure temperatures          & TEMPS         \\ \hline
c & Read clock in cycles          & CLOCK         \\ \hline
U & UNLOCK + STANDBY all ADS131A04& ADC\_REGS     \\ \hline
q & Read ADS131A04 registers      & ADC\_REGS     \\ \hline
e & Read measurement config       & CONFIG        \\ \hline
W & WAKEUP + LOCK, start measuring& SAMPLES [...] \\ \hline
S & Reboot                        & {\emph None}  \\ \hline
? & Print help                    & INFO          \\ \hline
{\textbackslash}x1B (ESC)  &   Stop measurement, abort current recvline().
Can occur anywhere in a line, does not have to be terminated with newline. & ESC \\ \hline
\end{tabular}
\caption{Command table, parameterless commands}
\label{command_table_0}
\end{centering}
\end{table}

\begin{table}[H]
\begin{centering}
\rowcolors{1}{lgray}{dgray}
\begin{tabular}{|p{0.8cm}|p{2.6cm}|p{4.9cm}|p{2.0cm}|}
\hline
{\bf Char} & {\bf Parameter syntax} & {\bf Description} & {\bf Response} \\ \hline
M & id pwm              & Set motor PWM                       & MTR\_PWM      \\ \hline
M & pwm0 pwm1 pwm2      & Set all motor PWMs                  & MTR\_PWM      \\ \hline
O & id vgnd             & Set VGND DAC value                  & VGNDS         \\ \hline
O & vgnd0 vgnd1 vgnd2   & Set all VGND DAC values             & VGNDS         \\ \hline
Q & id addr val         & Set ADS131A04 register              & ADC\_REGS     \\ \hline
E & frame\_size gap [num\_packets] & Configure measurement    & CONFIG        \\ \hline
  &                     & Read CPU regs (\$0000 - \$00FF)     &               \\ \hline
  &                     & Write CPU regs (\$0000 - \$00FF)    &               \\ \hline
  &                     & Read RAM (\$0100 - \$FFFF)          &               \\ \hline
  &                     & Write RAM (\$0100 - \$FFFF)         &               \\ \hline
  &                     & Read EEPROM (\$000 - \$FFF)         &               \\ \hline
  &                     & Write EEPROM (\$000 - \$FFF)        &               \\ \hline
  &                     & Read ROM (\$00000 - \$1FFFF)        &               \\ \hline
  &                     & Read fuses                          &               \\ \hline
C & cycles              & Set clock in cycles                 & CLOCK         \\ \hline
w & cycles              & Wait given number of cycles         & INFO          \\ \hline
  &                     & Start measurement                   &               \\ \hline
\end{tabular}
\caption{Command table, commands with parameters}
\label{command_table}
\end{centering}
\end{table}

\subsection{Short example}

To get the instrument into a useable state at least the following commands have to be issued:

\begin{itemize}
\item Start motors ('M')
\item RESET+UNLOCK+STANDBY ADS131A04s ('U')
\item Configure ADS131A04s ('Q')
\item Configure measurement ('E')
\item WAKEUP+LOCK ADS131A04s ('W')
\end{itemize}

What follows is a slightly edited session (for brevity) with only one fieldmill mounted.
First just the commands are listed, then the full session (input and output):

\begin{lstlisting}
    M1 1023     # Motor 1 at full speed
    U           # UNLOCK ADCs
    Q1 0F 01    # Activate channel 1 on ADC 1
    E100 0 3    # 100 frames per packet, no gap, 3 SAMPLES
    W           # WAKEUP ADCs, output SAMPLES
\end{lstlisting}

Full session:

\begin{lstlisting}
    BUSY
    *INFO
    Hello, Earth! 👽
    READY
    M1 1023     # Motor 1 at full speed
    BUSY
    *MTR_PWM
    0 1023 0
    READY
    U           # UNLOCK ADCs
    BUSY
    [...]
    *ERROR
    ADC 0 seems to be offline
    *INFO
    ADC 1 up
    [...]
    *ERROR
    ADC 2 seems to be offline
    *ADC_REGS
    0 ff ff ff ff ff ff ff ff ff ff ff ff ff ff ff ff ff ff ff ff ff
    1 04 03 00 00 00 00 00 01 00 00 00 60 3c 08 86 00 00 00 00 00 00
    2 ff ff ff ff ff ff ff ff ff ff ff ff ff ff ff ff ff ff ff ff ff
    READY
    Q1 0F 01    # Activate channel 1 on ADC 1
    BUSY
    *ADC_REGS
    0 ff ff ff ff ff ff ff ff ff ff ff ff ff ff ff ff ff ff ff ff ff
    1 04 03 20 00 00 01 00 01 00 00 00 60 3c 08 86 01 00 00 00 00 00
    2 ff ff ff ff ff ff ff ff ff ff ff ff ff ff ff ff ff ff ff ff ff
    READY
    E100 0 3    # 100 frames per packet, no gap, 3 SAMPLES
    BUSY
    *INFO
    bytes = 420, cpc = 25600, pc = 1
    cycles_out =       276172
     cycles_in =      2560000 (OK)
    *CONFIG
    100 0 3
    READY
    W           # WAKEUP ADCs, output SAMPLES
    BUSY
    *INFO
    Measurement started
    READY
    BUSY
    *SAMPLES
    [...]
\end{lstlisting}


\section{Responses}

Responses given before commands since many commands share response format.

\subsection{INFO}

Informative text.
Display contents in text window or terminal.
Should be displayed with normal colors.
Example:

\begin{lstlisting}
    BUSY
    *INFO
    Starting temperature conversion...
    READY
\end{lstlisting}


\subsection{WARNING}

Warning.
Should be displayed with a yellowish or maybe orange color.

\begin{lstlisting}
    BUSY
    *WARNING
    Instrument issues no warnings currently,
    but may in the future.
    READY
\end{lstlisting}


\subsection{ERROR}

Some kind of error.
Should be displayed with a reddish color.

\begin{lstlisting}
    BUSY
    *ERROR
    One or more of PWMS 1111, 2222, and 3333
    is greater than MOTOR_TOP = 1023
    READY
\end{lstlisting}


\subsection{VOLTAGES}

TODO

\subsection{MTR\_PWM}

PWM value for each of the three motors, as integers between 0 - 1023.
These correspond roughly to 0 - 6000 RPM, but not exactly.
For more information see EC20 datasheet.
Example response after issuing a "K" command:

\begin{lstlisting}
    BUSY
    *MTR_PWM
    511 511 511
    READY
\end{lstlisting}

\subsection{VGNDS}

10-bit DAC values for each MAX504 (0 - 1023), which are used for the input stages' virtual grounds (VGND).
The MAX504s are in bipolar configuration,
so a value of 0 gives -2.048 V out, 512 gives 0 V out and 1023 gives 2.044 V out.
In other words the output voltage is $-2.048 + value*0.004$.
See also Table 3 in the MAX504 datasheet.

The instrument is initialized so that all MAX504s output zero volts.
Example response to an 'o' command after initialization:

\begin{lstlisting}
    BUSY
    *VGNDS
    512 512 512
    READY
\end{lstlisting}

\subsection{MTR\_SPD}

Motor speeds in RPM, and the number of tachometer impulses detected.
For a sensible output at least two tachs must have been seen for each motor.
Exact format isn't finalized, so display output as if it were INFO for now.

\begin{lstlisting}
    BUSY
    *MTR_SPD
    Motor 0: 0 tachs (0 RPM)
    Motor 1: 1 tachs (0 RPM)
    Motor 2: 2 tachs (1234 RPM)
    READY
\end{lstlisting}

\subsection{ONEWIRE}

1-Wire devices.
One line per device containing a 64-bit hexadecimal ROM string (16 characters).

\begin{lstlisting}
    BUSY
    *ONEWIRE
    28d09948090000ec
    286a1a690900005e
    28ad7548090000c5
    READY
\end{lstlisting}

\subsection{TEMPS}

Temperature readings.
One line per DS18B20Z device with 64-bit hexadecimal ROM and decimal temperature values in degrees Celsius.
Range is -40..+125 with a resolution of 1/16 degrees Celsius.

\begin{lstlisting}
    BUSY
    *TEMPS
    28d09948090000ec 24.12
    286a1a690900005e 24.62
    28ad7548090000c5 -18.56
    READY
\end{lstlisting}

\subsection{CLOCK}

Current time in CPU cycles as a 64-bit decimal value.
May roll over in theory, but it would take 79,000 years unless the user
specifically set the clock to something close to the maximum value (18446744073709551615).

\begin{lstlisting}
    BUSY
    *CLOCK
    3702994144
    READY
\end{lstlisting}

\subsection{SAMPLES}

Sample data.
This response is binary.
It is possible, but very unlikely, that the binary data contains the string "READY{\textbackslash}r{\textbackslash}n".
To avoid trouble it is best to compute the exact number of bytes in the sample data by following listing \vref{sample_packet_s}.

\begin{lstlisting}
    BUSY
    *SAMPLES
    [sample_packet_s]
    READY
\end{lstlisting}

\subsection{ADC\_REGS}

The contents of all registers in all ADCs.
Three lines, each beginning with the integer ID of the ADC.
After this there are twenty-one (21) hexadecimal values corresponding to registers 00h - 14h.
See ADS131A04 datasheet for more information.

This example has one fieldmill installed in position 1.
Therefore only the middle line (ID=1) has valid data:

\begin{lstlisting}
    BUSY
    *ADC_REGS
    0 ff ff ff ff ff ff ff ff ff ff ff ff ff ff ff ff ff ff ff ff ff
    1 04 03 00 00 00 00 00 01 00 00 00 60 3c 08 86 00 00 00 00 00 00
    2 ff ff ff ff ff ff ff ff ff ff ff ff ff ff ff ff ff ff ff ff ff
    READY
\end{lstlisting}

The first two registers (ID\_MSB and ID\_LSB) can be used to detect that an ADC is working,
since they are read-only and of known value.
All our chips have ID\_MSB = 04h (for ADS131A{\emph 04}) and ID\_LSB = 03h (revision 3).

\subsection{CONFIG}

Measurement configuration. Three integers.
The first integer is the number of frames in each SAMPLES packet.
Zero means no configuration yet, or there was some error with the attempted configuration.
In short, if the first integer is zero then WAKEUP ('W') cannot be issued.
The second iteger is the gap in frames between SAMPLES packets.
It may be zero, and must be non-zero if the total sample bitrate is higher than the UART bitrate.
So a non-zero value provides a longer pause between SAMPLES packets.
The third integer is the number of SAMPLES packets that will be output.
A value of 65535 means SAMPLES will be output indefinitely.

The following example has 100 frames per packet, no gap between packets and will run indefinitely:

\begin{lstlisting}
    BUSY
    *CONFIG
    100 0 1000
    READY
\end{lstlisting}

\section{Commands}

\subsection{Reboot ('S')}

Takes no parameters and no prisoners.
Hangs the user program, inducing the WDT to cause a software reset.
After the reset the CPU enters the bootloader.
This allows a serial programmer to keep sending 'S' to get the device into a known state,
since the AVR109 bootloader merely responds to 'S' with 'AVRBOOT'.
If no programming is performed then the user program is re-entered after the bootloader wait time has elapsed (typically three seconds).

\begin{lstlisting}
    S
    S
    AVRBOOT[three second delay]BUSY
    *INFO
    Hello, Earth! 👽
    READY
\end{lstlisting}


\subsection{Set motor PWMs ('M')}

Set one or more OCR1x.
PWM values must be between 0 - 1023.

\subsubsection{Two parameter form (id pwm)}

Set PWM duty rate for motor with given ID (0..2). Example:

\begin{lstlisting}
    M1 800
    BUSY
    *MTR_PWM
    0 800 0
    READY
\end{lstlisting}

\subsubsection{Three parameter form (pwm0 pwm1 pwm2)}

Set PWM duty rate for all three motors in one go.

\begin{lstlisting}
    M200 400 600
    BUSY
    *MTR_PWM
    200 400 600
    READY
\end{lstlisting}

\subsection{Set VGND DAC values ('O')}

Set one or more inputs to the MAX504s.
Values must be between 0 - 1023.

\subsubsection{Two parameter form (id vgnd)}

Set DAC value for MAX504 with given ID (0..2). Example:

\begin{lstlisting}
    O1 900
    BUSY
    *VGNDS
    512 900 512
    READY
\end{lstlisting}

\subsubsection{Three parameter form (vgnd0 vgnd1 vgnd2)}

Set DAC value for all MAX504s:

\begin{lstlisting}
    M300 400 500
    BUSY
    *VGNDS
    300 400 500
    READY
\end{lstlisting}

\subsection{UNLOCK + STANDBY all ADS131A04 ('U')}

Bring ADCs out of reset, halting any ongoing conversions and allowing registers to be modified.

\subsection{Set ADS131A04 registers ('Q')}

The 'Q' command sets the value of one register in one of the ADCs.
After the 'Q the line must contain three values: an integer specifying the ADC ID,
a hexadecimal value specifying the register address and a hexadecimal value specifying the value.
In C formatting parlance it should be "\%i \%x \%x".

The following example sets COMP\_TH = 111b and enables fixed word size (6 device words per frame regardless of ADC\_ENA):

\begin{lstlisting}
    Q1 0B 67
    BUSY
    *ADC_REGS
    0 ff ff ff ff ff ff ff ff ff ff ff ff ff ff ff ff ff ff ff ff ff
    1 04 03 00 00 00 00 00 01 00 00 00 67 3c 08 86 00 00 00 00 00 00
    2 ff ff ff ff ff ff ff ff ff ff ff ff ff ff ff ff ff ff ff ff ff
    READY
    Q1 0C 3E
    BUSY
    *ADC_REGS
    0 ff ff ff ff ff ff ff ff ff ff ff ff ff ff ff ff ff ff ff ff ff
    1 04 03 00 00 00 00 00 01 00 00 00 67 3e 08 86 00 00 00 00 00 00
    2 ff ff ff ff ff ff ff ff ff ff ff ff ff ff ff ff ff ff ff ff ff
    READY
\end{lstlisting}

\subsection{Configure measurement ('E')}

Takes between two and four integer arguments.
The first integer sets the number of frames per packet.
The second integer sets the gap between packets.
The third integer, if present, sets the number of SAMPLES packets that will be captured and output (0..65534).
The fourth integer, if present, sets the sample format to one of the following:

\begin{itemize}
\item 0: Signed 24-bit samples
\item 1: Signed 8-bit samples. Full (quantized) 24-bit samples can be recovered by multiplying values by $2^{sample\_shift}$
\end{itemize}

Given values will be sanity checked.

\begin{lstlisting}
    E100 0 1000
    BUSY
    *INFO
    bytes = 420, cpc = 25600, pc = 1
    cycles_out =       276172
     cycles_in =      2560000 (OK)
    *CONFIG
    100 0 1000
    READY
    E10000 0
    BUSY
    *ERROR
    sample_data_size = 30000 larger than maximum 4096
    READY
    e
    BUSY
    *CONFIG
    0 0 65535
    READY
\end{lstlisting}


\subsection{Wakeup + LOCK, start measuring ('W')}

Sends WAKEUP and LOCK to all ADCs configured with ADC\_ENA != 0,
then outputs SAMPLES until the user sends ESC or 'U'.

\begin{lstlisting}
    W
    BUSY
    *SAMPLES
    [sample_packet_s]
    READY
    BUSY
    *SAMPLES
    [sample_packet_s]
    READY
    [...]
    ^[
    BUSY
    *ESC
    READY
\end{lstlisting}

\newpage
\section{Listing}
\label{sample_packet_s}

\lstinputlisting{../app/wire_structs.h}



\end{document}
