\documentclass{article}
\usepackage{listings}
\usepackage[table]{xcolor}
\usepackage{varioref}
\usepackage{float}
\begin{document}

\title{Hello}

\section{Greeting}
Hello world!



Culpa aliquid aperiam consequatur iste aut corrupti. Eos non harum dolor corrupti magnam consequuntur quod. Eos et quam culpa asperiores inventore ad distinctio sunt. Non in alias facilis quam quis vero quod dignissimos. Culpa ipsam velit quia. Ipsa eum enim repudiandae ut sunt doloribus.

Aut assumenda aut asperiores. Rerum ut nihil quisquam et. Quia et tempora vel. Illo excepturi soluta repellendus id debitis. Totam fuga doloribus exercitationem eius.

Et dolores amet id velit. Et ab saepe suscipit temporibus iusto. Odio dicta error voluptas veritatis ab ut aut doloribus. Et facere illo autem ea saepe accusantium eos minus.

Quis quidem porro tempora consequatur. Voluptatibus adipisci quibusdam illo magnam. Ut consequatur quia ipsum. Enim ea tenetur mollitia sed quaerat et neque officiis. Nostrum harum explicabo ut.

Culpa quia voluptas cumque officia. Impedit est ad et assumenda dolorem ex et. Voluptatem atque alias dicta ratione laborum modi rem.

\section{Commands}

All commands are human-readable, start with a single ASCII character and are termianted with a line ending.
If the command takes no parameters then the line ending is optional.
Comments may be added by inserting a hash sign ('\#').
This causes the rest of the line to be ignored (characters are consumed and discarded until end-of-line).
Parameter parsing is handled by sscanf(), which allows for the same command character to take a varying number
of parameters. An example of this is the 'M' command which exists in two-parameter and three-parameter forms:

\begin{lstlisting}
    M0 10       # Set speed of motor 0 to 10
    M10 10 10   # Set speed of all motors to 10
\end{lstlisting}


Line endings can either be carriage return ('{\textbackslash}r', ASCII code 13) or linefeed ('{\textbackslash}n', ASCII code 10), but never both in the same line.
In other words both Unix and Mac line endings are OK, but Windows line endings ("{\textbackslash}r{\textbackslash}n") are not.
This ensures both {\it echo} and {\it minicom} works as expected.
Output from the instrument is terminated by Windows line endings however, in order to work well with {\it minicom}.

Table \vref{command_table} summarizes all commands and their parameters.
More detailed descriptions of each command are given in the subsections that follow.

\definecolor{lgray}{gray}{0.95}
\definecolor{dgray}{gray}{0.90}

\begin{table}[H]
\begin{centering}
\rowcolors{1}{lgray}{dgray}
\begin{tabular}{|p{1.8cm}|p{1.8cm}|p{1.8cm}|p{5cm}|}
\hline
{\bf Command} & {\bf Parameter count} & {\bf Parameter syntax} & {\bf Description} \\ \hline
m & 0 &                     & Read motor speeds \\ \hline
M & 2 & ID spd              & Set motor speed \\ \hline
M & 3 & spd spd spd         & Set motor speeds \\ \hline
  & 0 &                     & Stop all motors \\ \hline
  & 1 & ID                  & Stop specific motor \\ \hline
  & 0 &                     & Measure motor speeds in RPM \\ \hline
  & 1 &                     & Measure specific motor speed in RPM \\ \hline
  &   &                     & Measure temperatures \\ \hline
  &   &                     & Configure ADC \\ \hline
  &   &                     & Read ADC configuration \\ \hline
  &   &                     & Read registers (\$0000 - \$00FF) \\ \hline
  &   &                     & Write registers (\$0000 - \$00FF) \\ \hline
  &   &                     & Read RAM (\$0100 - \$FFFF) \\ \hline
  &   &                     & Write RAM (\$0100 - \$FFFF) \\ \hline
  &   &                     & Read EEPROM (\$000 - \$FFF) \\ \hline
  &   &                     & Write EEPROM (\$000 - \$FFF) \\ \hline
  &   &                     & Read ROM (\$00000 - \$1FFFF) \\ \hline
  &   &                     & Read fuses \\ \hline
  &   &                     & Read clock \\ \hline
  &   &                     & Set clock \\ \hline
  &   &                     & Configure measurement (block size + gap) \\ \hline
  &   &                     & Read measurement configuration \\ \hline
  &   &                     & Start measurement \\ \hline
{\textbackslash}x1B (ESC)  &   &                     & Stop measurement \\ \hline
\end{tabular}
\caption{Command table}
\label{command_table}
\end{centering}
\end{table}

\subsection{Read motor speed ('m')}

Prints three integers containing OCR1A, OCR1B and OCR1C respectively.
0 - 255 roughly corresponding to 0 - 100\% or 0 - 6000 RPM.

\subsection{Set motor speed(s) ('M')}

Hardware will refuse if values are so low that they risk turning the motors off.

\section{Listing}

\lstinputlisting{sample_packet_s.h}



\end{document}
